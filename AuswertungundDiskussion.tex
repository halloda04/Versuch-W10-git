\section{Auswertung}
\subsection{Aufgabe 1, Zwickelabgleich}
Um den Wasserwert des Dewar-Gefäß zu ermitteln wird ein Zwickelabgleich durchgeführt.
Diagramm Zwickelabgleich
Der Wasserwert \omega wird durch die Formel
\begin{equation}
    \omega = \frac{m_W \cdot \left(T_W-T_m\right)}{T_m-T_K}-m_K
\end{equation}
berechnet.(Zitat Protokoll 1!!)Hierbei beschreibt $m_W$ das Gewicht 
des warmen Wassers, $T_W$ die Temperatur des Warmenwassers, $T_m$ die 
idialisierte Mischetemperatur des warmen und des kalten Wassers, 
$T_K$ die Temperatur des kalten Wassers und $m_K$ die Masse des kalten 
Wassers. Die jeweils gemessen Temperaturen konnten auf eine Genauigkeit 
von \Delta T \pm 0,05 °C bestimmt werden. Dadurch können die 
weiterführenden Geraden der Temperatur als Konstante Gerade verwendet 
werden. Das Gewicht des warmen Wassers konnte auf eine Genauigkeit von 
\pm 0,05g bestimmt werden $\left( m_W = 200,3 \pm 0,05g \right)$. Bei 
der Masse des kalten Wassers kam es zu einem Messfehler da beim 
schütten vom Becherglas ins Dewar-Gefäß etwas Wasser daneben gegangen 
ist, deswegen konnte hier nur eine Genauigkeit von \Delta 
$m_K = \pm 5g$ erzielt werden$\left(m_K = 185 \pm5g\right)$. Für den 
Fehler für $T_W$ und $T_K$ muss beachtet werden, dass der Zeitpunkt $t_0$ 
per Hand in das Diagramm eingefügt wurde und somit nicht den ideal 
Werten entspricht. Fehler der fitgeraden+ große formel!!