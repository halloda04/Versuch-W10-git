\section{Auswertung}
\subsection{Teil 1}

Der Wasserwert $w$ also die Menge an Wasser, welches die gleiche Wärmekapatzität hat,
wie das Dewar Gefäß, ist gegeben durch 
\begin{equation}
    w = \frac{m_w \,(T_w - T_M)}{(T_M - T_k)} - m_k
\end{equation}

Wobei $m_w = 200,5 g \pm 0,1$ = die Masse des warmen Wassers, $m_k = 185 \pm 3 $ die Masse des kalten Wasser und $T_w$, $T_M$ bzw $T_k$ Die Temperatur, die durch den Punkt an dem Gerade g(siehe unten) Die Ausgleichsgerade W bzw K schneidet.


