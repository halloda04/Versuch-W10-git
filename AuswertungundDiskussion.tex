\section{Auswertung}
\subsection{Aufgabe 1, Zwickelabgleich}
Um den Wasserwert des Dewar-Gefäß zu ermitteln wird ein Zwickelabgleich durchgeführt.
Diagramm Zwickelabgleich
Der Wasserwert \omega wird durch die Formel
\begin{equation}
    \omega = \frac{m_W \cdot \left(T_W-T_m\right)}{T_m-T_K}-m_K
\end{equation}
berechnet.(Zitat Protokoll 1!!)Hierbei beschreibt $m_W$ das Gewicht 
des warmen Wassers, $T_W$ die Temperatur des Warmenwassers, $T_m$ die 
idialisierte Mischetemperatur des warmen und des kalten Wassers, 
$T_K$ die Temperatur des kalten Wassers und $m_K$ die Masse des kalten 
Wassers. Die jeweils gemessen Temperaturen konnten auf eine Genauigkeit 
von \Delta T \pm 0,05 °C bestimmt werden. Dadurch können die 
weiterführenden Geraden der Temperatur als Konstante Gerade verwendet 
werden. Das Gewicht des warmen Wassers konnte auf eine Genauigkeit von 
\pm 0,05g bestimmt werden $\left( m_W = 200,3 \pm 0,05g \right)$. Bei 
der Masse des kalten Wassers kam es zu einem Messfehler da beim 
schütten vom Becherglas ins Dewar-Gefäß etwas Wasser daneben gegangen 
ist, deswegen konnte hier nur eine Genauigkeit von \Delta 
$m_K = \pm 5g$ erzielt werden$\left(m_K = 185 \pm5g\right)$. Für den 
Fehler für $T_W$ und $T_K$ muss beachtet werden, dass der Zeitpunkt $t_0$ 
per Hand in das Diagramm eingefügt wurde und somit nicht den ideal 
Werten entspricht. Fehler der fitgeraden+ große formel!!

\newpage
\subsection{Aufgabe 2, Wärmeleistung}
Trägt man die Temperatur $T_u$ gegen die Zeit $t$ auf so erhält man folgende Grafik:

\begin{figure}[H]
    \centering
    \includegraphics[width=\textwidth]{Bilder/tempuntent.png}
    \caption{Die Temperatur $T_u$ aufgetragen gegen die Zeit. Mit Grenz- und Ausgleichsgeraden. Die ersten 3 Messpunkte wurden in die Steigung der Geraden miteinbezogen da die Temperatur $T$ hier konstant ist}
    \label{fig:tempunten}
\end{figure}


Für die Wärme $Q$, welche vom Dewar-Gefäß aufgenommen wird gilt:

\begin{equation}
Q = c_w \cdot (m_w + w)\cdot \Delta T
\end{equation}

Differenziert nach der Zeit, ergibt sich für die Wärmeleistung $P$:
\begin{equation}
P = c_w \cdot (m_w + w)\cdot \frac{\text{d}T}{\text{d}t} 
\end{equation}

Für den Fehler gilt:

\begin{equation}
\Delta P = \pm \left(
\left|\frac{\partial P}{\partial m_w}\right| \,\Delta m_w
+ \left|\frac{\partial P}{\partial w}\right| \,\Delta w
+ \left|\frac{\partial P}{\partial \frac{dT}{dt}}\right| \,\Delta\!\frac{dT}{dt}
\right)
= \pm P \left(
\frac{\Delta m_w + \Delta w}{m_w + w}
+ \frac{\Delta\!\frac{dT}{dt}}{\frac{dT}{dt}}
\right)
\end{equation}

Dabei ist $\Delta\!\frac{dT}{dt}$ gegeben durch die differenz der Steigung der Grenzgeraden.

\begin{equation}
\Delta\!\frac{dT}{dt}
= \pm \left| \frac{
\left(\frac{dT}{dt}\right)_{\max}
-
\left(\frac{dT}{dt}\right)_{\min}
}{2} \right|.
\end{equation}

Die Menge and destilliertem Wasser war $m_w = 409,5 \pm 0,1g $. Für $c_W$ wurde eine Wert von $4,18\frac{\text{J}}{{\text{kgK}}}$ verwendet. Somit ergibt sich mit den Werten für die Steigungen aus \ref{fig:tempunten}
und dem in Teil 1 bestimmtem Wasserwert eine Wärmeleistung von:
\begin{equation}
P_{Cu} = FEHLer
\end{equation}

Für den Aluminuimstab sieht die Grafik wie folgt aus:

\begin{figure}[H]
    \centering
    \includegraphics[width=\textwidth]{Bilder/tempuntentalu.png}
    \caption{Die Temperatur $T_u$ aufgetragen gegen die Zeit. Mit Grenz- und Ausgleichsgeraden. Die ersten 4 Messpunkte wurden in die Steigung der Geraden miteinbezogen da die Temperatur $T$ hier konstant ist}
    \label{fig:tempuntenalu}
\end{figure}