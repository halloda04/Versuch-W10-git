\section{Einleitung}

Die Wärmeleitung ist ein grundlegendes Prinzip der Physik 
und sie kommt in allen möglichen Bereichen des Lebens vor, 
sowohl im täglichen Leben als auch in der Wissenschaft. Im 
täglichen Leben tritt sie bei so simplen Dingen wie heißem 
Tee oder auch beim Kochen auf, und in der Wissenschaft bei 
den Niedrigtemperatur-Mikroskopen der Experimentalphysik 5 
und bei den Plasmaexperimenten der Plasmaphysik in unserem 
Institut. Wärme ist ein wirklich alltägliches Phänomen, das 
bei eigentlich allem anfällt, was irgendwo passiert. Je 
genauer die Leitfähigkeit eines Stoffes bekannt ist, desto 
besser kann also Technik, die sich mit dem Transport von 
Wärme beschäftigt, wie Kühlgeräte oder Klimaanlagen, 
weiterentwickelt werden. Der Versuch ist dem Gebiet der 
Wärmelehre zuzuordnen. Dementsprechend sind mögliche 
Messgrößen mit Thermomether und Stoppuhr feststellbar.
