\section{Theoretische Grundlagen}


Beim Transport von Wärmeenergie gibt es nur 
drei unterschiedliche Phänomene. Das erste ist die 
Wärmestrahlung; hierbei wird Wärmeenergie über 
elektromagnetische Wellen, meist im Infrarotbereich, 
transportiert. Dies ist jedoch sehr ineffizient im Vergleich 
zu den anderen Arten des Wärmetransports. Aufgrund dessen 
ist es ziemlich komplex, vernünftige Kühlsysteme innerhalb 
eines Vakuums zu entwickeln. Die zweite Art des 
Wärmetransports nennt sich Konvektion; diese beschreibt die 
Bewegung von Gasen und Flüssigkeiten, die sich aufgrund von 
Temperaturschwankungen innerhalb eines Raumes bewegen. Die 
dritte und letzte Art des Wärmetransports ist die 
Wärmeleitung. Die Wärmeleitung funktioniert aufgrund von 
Stößen der im Material vorhandenen Atomkerne und Elektronen. 
Dieser Energietransport funktioniert sehr effizient 
innerhalb von Metallen, da in diesen die Elektronen sehr 
beweglich sind und somit viele Stöße passieren können, 
oder in Materialien, bei denen die Atomkerne sehr nah 
beieinander liegen.\\

In diesem Versuch wird die Wärmeleitung näher betrachtet; sie wird mathematisch durch folgende Gleichung ausgedrückt:\\

\begin{equation}
\frac{\partial T}{\partial t} = \frac{\lambda}{c \rho} \nabla^2 T
\end{equation}

In dieser Gleichung ist $T$ die Temperatur, $\lambda$ die 
Wärmeleitfähigkeitskonstante, $c$ die spezifische 
Wärmekapazität des Materials und $\rho$ die Dichte des 
untersuchten Materials. Im einfacheren eindimensionalen Fall gibt es eine kleine Abwandlung der Formel:\\

\begin{equation}
\frac{\partial T}{\partial t} = \frac{\lambda}{c \rho} \frac{\partial^2 T}{\partial x^2}
\end{equation}
