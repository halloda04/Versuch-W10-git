\section{Versuchsbeschreibung}

\subsection{Versuchsaufbau Teil 1}
Im ersten Teil wird ein zunächst ein Zwickelabgleich 
duchgeführt, um den Wasserwert des verwendeten Dewar-Gefäß 
zu bestimmen. Um diesen Zwickelabgleich durchzuführen 
werden, ein Dewar-Gefäß, zwei Messbecher wovon einer 
Teilweise Isoliert ist, zwei Thermometer, warmes und kaltes 
Wasser benötigt. Um den Vorgang zu erleichtern braucht man 
zusätzlich noch einen Magnetrührer und einen Magnetischen 
Rührfisch.
\subsection{Versuchsdurchführung Teil 1}
Nun zur Durchführung des Zwickelabgleiches. Hierfür werden 
180 gramm kaltes Wasser und 200g warmes Wasser benötigt. 
Das kalte Wasser wird direkt in das Dewar-Gefäß gegeben, 
und das warme Wasser in das teilweise isolierte Becherglas. 
Damit im zweiten Teil der Messung die Werte nicht verfälscht 
werden wird nun der Magnetrührer in Position gebracht und 
schonmal angeschaltet. In beiden Behältern wird nun für 5 
Minuten in 30 sekunden Abständen die Temperatur gemessen. 
Sobald die 5 Minuten vorbei sind wird das warme Wasser 
langsam in das kalte Wasser geschüttet, dieser Vorgang 
sollte ungefähr 1 bis 2 Minnuten dauern, währenddessen 
misst der Laborpartner alle 5 sekunden die Temperatur im 
Dewar-Gefäß. Sobald die beiden Wassermengen vermischt sind 
und der Temperatur verlauf wieder linear ist, wird der 
Temperatur Messrhythmus wieder von 5 sekunden auf 30 
sekunden erhöht diese letzte Phase der Temperatur Messung 
geht wieder 5 Minuten lang. 
\subsection{Versuchsaufbau Teil 2\&3}
hallo
\subsection{Versuchsdurchführung Teil 2\&3}
hallo