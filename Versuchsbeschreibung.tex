\section{Versuchsbeschreibung}

\subsection{Versuchsaufbau Teil 1}
Im ersten Teil wird zunächst ein Zwickelabgleich 
durchgeführt, um den Wasserwert des verwendeten Dewar-Gefäßes 
zu bestimmen. Um diesen Zwickelabgleich durchzuführen, 
werden ein Dewar-Gefäß, zwei Messbecher, wovon einer 
teilweise isoliert ist, zwei Thermometer sowie warmes und kaltes 
Wasser benötigt. Um den Vorgang zu erleichtern, braucht man 
zusätzlich noch einen Magnetrührer und einen magnetischen 
Rührfisch.

\subsection{Versuchsaufbau Teil 2\&3}
Der zweite Teil des Versuchs beschäftigt sich nun mit der 
Wärmeleitung. Für den Aufbau braucht man einen isolierten 
Kupferstab mit drei Löchern: eines an der Spitze und die 
anderen beiden jeweils an der oberen und unteren Kante der 
Isolierung, einen Lötkolben, zwei Thermometer, das 
Dewar-Gefäß mit dem Magnetrührer aus Teil 1 des Versuchs, 
eine höhenverstellbare Halterung, um den Kupferstab 
befestigen zu können, eine Stoppuhr und ein Thermometer. 
Als Erstes wird das leere Dewar-Gefäß mit dem Magnetrührer 
ausgerüstet, leer gewogen, befüllt und noch einmal gewogen, um 
das Leer- und Vollgewicht zu bestimmen. Als Nächstes wird der 
Kupferstab so über dem Dewar-Gefäß befestigt, dass er 
ungefähr zwei Zentimeter in das Wasser hineinragt. Danach 
wird der Lötkolben in dem Loch an der Spitze des 
Kupferstabes befestigt und die beiden Thermometer in den 
Löchern über und unter der Isolierung ebenfalls.  
Teil 3 des Versuchs hat einen analogen Aufbau; dort wird jedoch 
der Kupferstab durch einen Aluminiumstab ersetzt.

\newpage

\subsection{Versuchsdurchführung Teil 1}
Nun zur Durchführung des Zwickelabgleichs. Hierfür werden 
180 Gramm kaltes Wasser und 200 g warmes Wasser benötigt. 
Das kalte Wasser wird direkt in das Dewar-Gefäß gegeben 
und das warme Wasser in das teilweise isolierte Becherglas. 
Damit im zweiten Teil der Messung die Werte nicht verfälscht 
werden, wird nun der Magnetrührer in Position gebracht und 
schonmal angeschaltet. In beiden Behältern wird nun für 5 
Minuten in 30-Sekunden-Abständen die Temperatur gemessen. 
Sobald die 5 Minuten vorbei sind, wird das warme Wasser 
langsam in das kalte Wasser geschüttet; dieser Vorgang 
sollte ungefähr 1 bis 2 Minuten dauern. Währenddessen 
misst der Laborpartner alle 5 Sekunden die Temperatur im 
Dewar-Gefäß. Sobald die beiden Wassermengen vermischt sind 
und der Temperaturverlauf wieder linear ist, wird der 
Temperaturmessrhythmus wieder von 5 Sekunden auf 30 Sekunden 
erhöht. Diese letzte Phase der Temperaturmessung 
dauert erneut 5 Minuten.

\subsection{Versuchsdurchführung Teil 2\&3}
Sobald der Versuchsaufbau vollendet ist, kann mit der Durchführung 
begonnen werden. Hierzu werden der Magnetrührer und der 
Lötkolben angeschaltet und für 40 Minuten alle 60 Sekunden 
die Temperaturen beider Thermometer gemessen. Für 
Teil 3 wird nur der Kupferstab durch einen Aluminiumstab 
getauscht; die restliche Durchführung ist analog.
